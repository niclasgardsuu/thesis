
In this section, the results are discussed from the context of the purpose and goals which were set at the beginning of the project. 

\subsection{Discussing the Results}
The results from this thesis show us that the new allocators do not add any significant execution time that decreases the performance of the garbage collector, which is a good sign. Results also show that there is indeed an increased amount of operations being done when the free-list allocator is being used. However, the difference is not large enough to cause any performance issues for the Java program being executed. 

Another positive result is that some benchmarks used for evaluating the performance displayed good signs of utilizing the fragmented memory for compaction. The benchmark \textit{fop} performed relocations of objects that were able to use very much of the memory that was found in free-list as a location to compact objects into. With an average fragmentation of approximately 10\% when using the TLSF allocator, it shows us that the fragmented memory between objects is indeed usable. 

The fact that the execution time does not change is a good result in the sense that the expectation was to have a more computationally heavy allocation method, with the upside of being able to utilize memory more efficiently. The results also prove that the fragmented memory was indeed being utilized quite efficiently. However, these results could suggest that the allocator is not doing as much work as expected. The results from this thesis do not tell us enough about the behavior of the garbage collector to know if the amount of memory that was saved from using the fragmented memory was enough to cause any performance increases. Rather, the results hint towards the opposite that the memory saved is not enough to significantly change the performance of the program.

From the results about utilization, there is a noticeable problem. A very large amount of the free-list memory that is created every garbage collection cycle is never used. This is largely because of the method used to choose which pages should construct a free-list, because it currently does not try to balance the relocation set size with the amount of free space in the rest of the pages. However, the problem is not only due to the relocation set size but also the distribution of objects of different ages. Since the age of an object is page-dependent, a free-list can only contain objects of one unique age. It could be the case that free-lists are constructed for a target age that has no relocated objects, meaning the method for selecting a relocation set is unfit for utilizing the full potential of free-lists. 

\subsection{Discussing the Evaluation Method}
The method used for evaluating the performance of the free-lists provided valuable insights into how ZGC behaved with and without the presence of a free-list allocator. The difference between the two allocators, about the reference version of ZGC was also noticed to have an impact on certain metrics, which was one of the goals of evaluating both allocators.

While the choice of benchmarking programs did provide unique relocation patterns that display different performances under different conditions, there are certain issues with how the performance measurements were gathered. The main problem was that some programs did not have enough data points to be able to draw any conclusions. For example, \textit{xalan} from the Dacapo benchmark did not utilize the free-lists nearly as much as \textit{avrora}, as can be seen in Section~\ref{sec:results:frag} on Figure~\ref{fig:memory-fragmentation}, from the small amount of data points in the boxplot. The reason for this is that \textit{xalan} is a program that uses very little memory and does not create a lot of fragmentation when executing inside of the JVM with the chosen configurations. This makes it an unfit program for evaluating the usage of free-lists, since it does not impact the execution of the program much at all. By more thoughtfully picking out benchmarking programs, more valuable data could be gathered about the free-list utilization.

%%% Local Variables:
%%% mode: latex
%%% TeX-master: "main"
%%% End:
