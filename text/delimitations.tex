%%% Local Variables:
%%% mode: latex
%%% TeX-master: "main"
%%% End:
This report only covers the integration of a free list allocator in ZGC, and not the implementation of the allocator itself. The integration process assumes that such an allocator is already available, and the project focuses on integrating it into the ZGC code base.

As this project covers the integration of a free list allocator in a large code base, the project is limited to using the free-list based allocator in the compaction phase of the garbage collector when moving live objects. The garbage collector could potentially benefit from using a free-list based allocator for initially allocating objects, but the design of ZGC currently only performs those types of allocations on newly created regions of memory, meaning there are no fragmented holes.

The way ZGC classifies object sizes puts a delimitation on where the new relocation strategy will be applied. In order to get the most variation of the fragmented memory, objects classified as small will be targeted. Most allocations are small and tend to not be allocated for very long. This makes small allocations the prime subject for benefitting the most from constructing free-lists, since their short life span causes memory to get fragmented quicker. Properly evaluating the relocation of small allocations will be a better use of time than implementing the strategy for the other size classes.

Another delimitation to this project is related to how ZGC is a generational garbage collector, meaning it differentiates between \textit{young} and \textit{old} objects. This project will only cover the implementation of compacting young objects since those are the most common types of objects. Due to time constraints, it would be impractical to implement and evaluate the relocation strategy for old objects.