This thesis has presented a version of ZGC that offers the possibility of utilizing fragmented memory using free-lists. While this implementation has proved that it is possible to perform relocations using free-lists, there is a lot of work that has to be done to reach the full potential of this new relocation technique. This section proposes some future work that could lead to improvements in the utilization of free-lists. Also proposed in this section are some recommended future work that aims to solve problems with the solution presented in this thesis.

\subsection{Potential Future Work}
In the HotSpot VM, the current minimum size of an object is 16 bytes, which forces objects smaller than this to use more memory than necessary. This results in internal fragmentation, which allocates more memory than the program needs. In parallel to this thesis, a project called \textit{Lilliput} is being pursued, which explores the possibility of decreasing the minimum size of objects to 8 bytes from 16~\cite{lilliput}. While this will automatically lower the total amount of memory used by Java programs, it could also prove beneficial when using free-lists. By allowing free-lists to be represented by an even higher degree of granularity, a larger amount of fragmented memory can be represented by the blocks in the free-list.

An additional path worth exploring is to look into the possibility of handling the age of objects on a per-object basis, such that the object itself can tell how old it is, and not base it on the age of the page that it is in. For every object in the HotSpot VM, a 32-bit header of metadata is allocated called the \textit{markword}. The markword contains 4 bits that ZGC currently does not make use of. 4 bits can represent $2^4 = 16$ different states, the same as the amount of different ages that objects can have in ZGC. In the current version of ZGC, mutator allocations would not benefit from utilizing free-lists. This is because no objects of the initial \textit{eden} age are ever determined to be garbage until the first GC cycle, and hence there cannot exist a free-list representation of any fragmented memory. By allowing \textit{eden} objects to exist with objects of other ages, mutator allocations could be done in memory with existing fragmentation, and could potentially allow for the use of free-lists in a new scenario.

\subsection{Recommended Future Work}
The work done in this thesis has not changed how ZGC chooses the relocation set. Pages are chosen to be part of the relocation set depending on the number of live objects that need to be evacuated, which is how ZGC reclaims the fragmented memory in that page. This method of choosing the relocation set is based on the fact that the unused data in pages was not reachable, and could only be utilized by evacuating the entire page. The introduction of a free-list changes this assumption, and opens up the possibility of using the memory without evacuating the page. From the results of this thesis, we can see that the relocation set that is selected is not fit for this use, since the amount of memory represented by free-lists overshoots the size of the relocation set by a lot. By predicting the size of the relocation set, and how much memory is being relocated for each age, free-lists can be constructed to better match the size of the relocation set. This way, ZGC would avoid spending as much time constructing free-lists that will never be used.

Another way forward is to design a high-level allocator that can represent a free-list across multiple pages. In this thesis, the allocators are used on a per-page basis. If an allocator fails to allocate an object, the page is deemed exhausted and the free-list of that page becomes unusable. From the results of this thesis, it could be seen that some benchmarking programs struggled with fragmentation in exhausted free-lists. With the use of a high-level allocator that can combine all pages into a single large free-list, it would not be necessary to exhaust pages to start using another page's free-list, reducing the fragmentation caused by failed allocations.

%%% Local Variables:
%%% mode: latex
%%% TeX-master: "main"
%%% End:
