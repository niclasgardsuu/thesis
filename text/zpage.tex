When using the ZGC, memory is stored in regions, which is referred to as pages. Pages are classed as \textit{Small}, \textit{Medium} or \textit{Large}. When new objects are allocated, they are allocated inside of these pages. An illustration of how the allocation of new objects in pages, as well as garbage collection inside of the page is shown in Figure~\ref{fig:zpages}.

%bild1 tom page
%bild2 page med 2 allokeringar
%bild3 pafge med en av allokeringarna markerad som LIVE och en DEAD
%bild4 page som visar att de är external fragmentation
\begin{figure*}
    \centering
    \begin{subfigure}[b]{0.475\textwidth}
        \centering
        \includesvg[scale=0.4]{figures/zpage_empty}
        \caption[Network2]%
        {This image shows the state of a newly allocated Z-Page where no objects have yet to be allocated inside it.}    
        \label{fig:page:empty}
    \end{subfigure}
    \hfill
    \begin{subfigure}[b]{0.475\textwidth}  
        \centering 
        \includesvg[scale=0.4]{figures/zpage_allocated}
        \caption[]%
        {This image shows a scenario where two objects of size 64B and 128B respectively have been allocated to the empty Z-Page.}    
        \label{fig:page:allocated}
    \end{subfigure}
    \vskip\baselineskip
    \begin{subfigure}[b]{0.475\textwidth}   
        \centering 
        \includesvg[scale=0.4]{figures/zpage_liveness}
        \caption[]%
        {This image represents the aftermath of doing a liveness check after the pointer to the first object has been lost. This means that the first object is considered dead, but the second object is still alive.}    
        \label{fig:page:liveness}
    \end{subfigure}
    \hfill
    \begin{subfigure}[b]{0.475\textwidth}   
        \centering 
        \includesvg[scale=0.4]{figures/zpage_fragmented}
        \caption[]%
        {After the garbage collection cycle, if the page was not relocated due to other pages being prioritized the page ends up with externally fragmented memory. The 64B at the top of the page will not be usable.}    
        \label{fig:page:fragmented}
    \end{subfigure}
    \caption[]
    {A representation of the allocated memory inside of a Z-Page} 
    \label{fig:zpages}
\end{figure*}
As can be seen in Figure~\ref{fig:page:fragmented}, there is going to some external fragmentation because of how the pages make use of the bump pointer in order to allocate new objects, effectively meaning the freed object at the top is still allocated because the program is no longer able to allocate new objects in its place. To counter this, and make the memory usable again, the ZGC uses relocation. Relocation is done by moving all of the live objects from one page to other pages. Three different things might happen here:\\
\begin{enumerate}
    \item There are pages with usable memory: This means that the objects that are going to be relocated can be placed in pre-existing pages, and the evacuated page can be freed entirely. 
    \item All pages are full: When all pages are full, the objects cannot simply be moved to other existing pages, but instead have to allocate an entirely new page. This new page is where the objects are going to be evacuated to. This removes fragmentation because the new page will have stored all of the objects in a continuous manner.
    \item The heap is full: When there are no pages with free memory, and there is no room to allocate new pages, the only way to free up space is by performing an in-place compaction. This effectively means the objects are being replaced inside of the current page that they are in. This will compact the page and remove any external fragmentation, at the cost of being a very expensive operation.
\end{enumerate}

In Figure~\ref{fig:relocation_scenarios} all three scenarios are illustrated by examples of what the heap may look like with some amount of pages with objects allocated inside them.

\begin{figure}
    \centering
    \includesvg[scale=0.5]{figures/zpages_relocation}
    \caption{Illustration of the three scenarios of relocation.}
    \label{fig:relocation_scenarios}
\end{figure}
%%% Local Variables:
%%% mode: latex
%%% TeX-master: "main"
%%% End:
