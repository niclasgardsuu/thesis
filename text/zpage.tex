ZGC is a region-based garbage collector, which means it allocates objects in regions. In ZGC, these regions are referred to as pages. Pages can have three different types: \textit{Small}, \textit{Medium} or \textit{Large}, with each type of page having different sizes. The sizes of the different types are illustrated in Table~\ref{table:zpage_sizes}. When new objects are allocated, they are allocated inside these pages. While the smalles allocation size is 16, it is still possible for mutator threads to request smaller amounts of memory. There will always be a multiple of 16 bytes of memory allocated since the accuracy of the pointers will not allow for smaller allocations.

The reason why the smallest allocation size is 16 Bytes is because of internal fragmentation. Due to the accuracy of pointers used in the ZGC, smaller allocations would lead to non-injective memory allocations. An example of what a Z-Page might look like, as well as the overhead data that is available for the garbage collector, is shown in Figure~\ref{fig:zpages}. There are some key aspects of the page that allow the garbage collector to efficiently do allocations, as well as work concurrently. The following important class members exist, and are labeled in the figure:
\begin{itemize}
    \item {\textbf{Bump Pointer}} \\
          The bump pointer is a pointer that exists within the page's memory, and lets the garbage collector know where the next allocation will happen. The bump pointer is moved forward every time an object is allocated in the page, and will therefore guarantee that any space from the bump pointer to the end of the page's memory will be available for allocation. In Figure~\ref*{fig:zpages}
    \item {\textbf{Live Map}} \\ 
          The live map is used by the garbage collector to keep track of which allocations are currently live. On the right hand side of Figure~\ref*{fig:zpages}, the livemap is shown. As the garbage collector is a mark-and-compact collector, the live map is not used to identify which objects to free from memory, but instead identify which objects to relocate to other memory pages.
    \item {\textbf{Sequence Number}} \\
          Every page has a sequence number which denotes during which garbage collection cycle a page was created. New allocations are only done in pages that have been created after the latest garbage collection cycle. Pages which are created before the garbage collection cycle are therefore separated from the new ones, allowing the garbage collector to work concurrently while mutator threads can still allocate new objects. For example, if the current Java program has performed a garbage collection 5 times, any allocations done after that will only be done on pages with the sequence number 5. And if the program decides to run a 6th garbage collection, only garbage will be collected from pages with sequence number 0-5, even though there might be concurrent threads allocating new objects to new pages with sequence number 6.
\end{itemize}

\begin{table}[H]
\centering
\begin{tabular}{lllll}
\hline
Page Type & Page Size          & \multicolumn{3}{l}{Object Size}        \\ \hline
Small     & 2 MB                & \multicolumn{3}{l}{{[}16B, 256KB{]}}   \\
Medium    & 32 MB               & \multicolumn{3}{l}{(256KB, 4MB)}       \\
Large     & $\geq$ 4 MB & \multicolumn{3}{l}{$\geq$ 4MB} \\ \hline
\end{tabular}
\caption{Page sizes in ZGC. (Figure taken from~\cite{zpage_size_table}). }
\label{table:zpage_sizes}
\end{table}

\begin{figure*}
    \centering
    \includesvg[scale=0.8]{figures/zpage_members}
    \caption[]
    {An example of a Z-Page and the overhead data that is used by ZGC.} 
    \label{fig:zpages}
\end{figure*}

\subsubsection{The GC Cycle}
In Figure~\ref{fig:zgc_timeline} a timeline of all the allocations on the heap is shown in terms of Z-Pages. The timeline shows the different phases of a garbage collection cycle, and how the garbage collector prepares for relocating memory in order to free unused memory.

\begin{figure}[!h]
    \centering
    \includesvg[scale=0.5]{figures/ztimeline}
    \caption{An overview of the heap}
    \label{fig:zgc_timeline}
\end{figure}

\begin{enumerate}
    \item In Figure~\ref*{fig:zrel1}, the initial state of the heap before a garbage collection is shown. One page has about 25\% free memory, while the other has about 50\%. The current GC cycle is 1, as shown by the top left digit of each page.
    \item In Figure~\ref*{fig:zrel2}, the GC cycle has started and the old pages have been frozen. This means that any concurrent allocations will be done in new pages. This is represented by the blue frozen pages, and the new gray page with a new sequence number.
    \item In Figure~\ref*{fig:zrel3}, the GC has finished the liveness marking of the objects located in the first two pages. 
\end{enumerate}

This will end up in a relocation of the three objects inside the first page. This could end in two different ways, shown in Figure~\ref{fig:zreloc}.

\begin{figure}[!h]
    \centering
    \includesvg[scale=0.5]{figures/zreloc_cases}
    \caption{An overview of the heap}
    \label{fig:zreloc}
\end{figure}
For case \textbf{A)} the relocation finds a suitable spot for a new page, in which the objects will be relocated to. The new page will store the objects contiguously in memory, and will allow the GC to free the previous page entirely.

For the other case \textbf{B)}, the heap is full and there is no more room for a new page. This means that in order to move the objects, the objects will have to be moved inside the page that they are already in. This is referred to as an in place compaction. An in place compaction is very computationally expensive, since it requires the thread to block every other concurrent thread who tries to alter the page. 
%%% Local Variables:
%%% mode: latex
%%% TeX-master: "main"
%%% End:
