
%%% Local Variables:
%%% mode: latex
%%% TeX-master: "main"
%%% End:
%The goal of this thesis project is to investigate the possibility of using free-list allocators in garbage collectors. The project will focus on integrating a free-list allocator in the ZGC, a garbage collector available in the most recent release of the OpenJDK. Currently in ZGC, memory is allocated in regions using sequential allocators, discussed in more detail in Section~\ref{sec:background}. Although it is a very performance effective way of allocating objects, it is known that bump pointers can cause a lot of external fragmentation in the memory~\cite{TODO:bump}. By making use of a free-list allocator, it is possible to let the garbage collector know that there is free space available in between allocations, allowing for allocations inside of the externally fragmented memory. This will add some extra work load on the allocation of objects, but will allow for using memory more efficiently. The project will investigate the feasibility of using such an allocator, and whether it improves the throughput and or memory usage.

% GAMMAL 1: Garbage collection is a feature in many programming languages with dynamic memory allocations, where the task of a garbage collector is to keep track of which parts in the memory is currently being used by the running program. This makes it possible for the garbage collector to automatically free up unused memory allocations. Efficient memory management is useful for increasing the performance and resource usage of programs, since all of the memory utilization is heavily relying on how the memory is being managed by the underlying garbage collector.

The most common technique for automatic memory management is garbage collection. The task of a garbage collector is to keep track of which parts of memory are being used by the program which is being executed. Different implementations of garbage collectors offer different benefits over others. The choice of garbage collector heavily impacts the performance and characteristics of the program being executed.

Java is a programming language with dynamic memory management that makes use of garbage collectors in order to handle memory. Operating within a runtime environment known as the Java Virtual Machine (JVM), Java allows users to configure the JVM to improve performance, including the choice of garbage collector. This thesis explores possible improvements that can be made to the JVM and its garbage collectors. Specifically, the goal is to look at the possibility of implementing a new method for compacting memory in ZGC, one of Java's most recently added garbage collectors.

ZGC stores data in different regions of memory, where every region includes a set of objects that has been allocated by the Java program running. During the compaction phase of a garbage collector, the goal is to decrease the amount of fragmented memory in these regions. A common issue with garbage collectors is the presence of fragmented memory. Fragmented memory can occur when previously allocated object are suddenly not needed by the program. These dead objects then create holes of unused memory which lie inbetween other live objects still used by the Java program. 

In ZGC, a so called \textit{bump pointer} is used for allocating objects, which is offers a fast method of allocating objects, but limits the choice of where to place the objects. In order to reclaim the unreachable memory, ZGC performs compaction on a region by evacuating all of its live objects into a new empty region, placing them more compactly in order to get rid of the holes in memory. And if there is no available memory to create an empty region, a more computationally heavy operation will be done to compact all objects into the same region they are currently in. Creating new regions and doing computationally heavy operations in order to perform compaction are the only two options for ZGC, but with the new method that is explored in this thesis, it is possible to relocate objects straight into the fragmented holes of memory using a free-list based allocator. The benefit of representing memory with free-lists is that it does not limit the available memory to a singular contiguous block of memory like the bump-pointer, but 

During this thesis, two different free-list based allocators were integrated into ZGC. Both allocators offer the same functionality, but differ in their performance due to their different design choices. J. Sikström, and C. Norrbin have both conducted separate thesises on the implementation of a TLSF allocator~\cite{JOELS:TODO} and a Buddy allocator~\cite{CASPERS:TODO}, respectively. These allocators were designed by Sikström and Norrbin to meet the requirements of ZGC, and were then integrated by me into ZGC to be used for compacting memory. 