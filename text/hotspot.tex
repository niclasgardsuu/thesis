
A virtual machine (VM) emulates a distinct computer in order to run various programs, which could be anything from full operating systems - to more specialized machines. The Java Virtual Machine (JVM)~\cite{JVM} is a specification for a virtual machine designed to execute Java programs. The JVM translates the program to instructions for the underlying machine, creating an abstraction of the hardware of the physical machine and allows Java programs to be run anywhere, as long as a JVM exists for that specific platform. 

OpenJDK~\cite{openjdk} is a set of tools for creating and running Java programs, maintained by Oracle. HotSpot~\cite{hotspot} is one of these tools, and is the reference implementation of the JVM. HotSpot is compromised of several parts for running Java applications, such as an interpreter, a Just-In-Time (JIT) compiler, and a garbage collector (GC). In combination, these components provide the means for running different types of Java programs on the platforms supported by HotSpot.

%%% Local Variables:
%%% mode: latex
%%% TeX-master: "main"
%%% End:

