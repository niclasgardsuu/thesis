A virtual machine (VM) is software that emulates a distinct computer in order to run various programs that could be anything from full operating systems - to more spacialised machines. The Java Virtual Machine (JVM)~\cite{JVM} is a specification for a specific virtual machine designed to execute Java programs. The JVM translates the program to instructions for the underlying machine, creating an abstraction of the hardware of the physical machine and allows Java programs to be run independently, as long as a JVM exists for that specific platform. 

One implementation of the JVM is HotSpot~\cite{hotspot}. HotSpot is the reference implementation of the JVM and is maintained by Oracle. HotSpot is compromised of several parts needed for running Java applications, such as an interpreter, a Just-In-Time (JIT) compiler, and a garbage collector (GC). These components collaborate to create an efficient and performant solution for running different types of Java programs.

%%% Local Variables:
%%% mode: latex
%%% TeX-master: "main"
%%% End:
