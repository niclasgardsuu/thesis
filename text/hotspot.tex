A virtual machine (VM) is software that emulates a different computer in order to run various programs. This could be anything from full operating systems, to smaller, program-specific machines. The Java Virtual Machine (JVM) \cite{JVM} is a specification for a specific virtual machine, designed to execute Java programs. The JVM translates the program to instructions for the underlying machine. This abstracts the hardware of the physical machine, and allows Java programs to be run independently, as long as a JVM exists for that specific platform. 

One such implementation of the JVM is HotSpot \cite{hotspot}. HotSpot is the reference implementation of the JVM, maintained by Oracle. It contains many parts needed for running Java applications, such as an interpreter, a Just-In-Time (JIT) compiler, and a garbage collector (GC). All components work in tandem to create a performant solution for running different types of Java programs efficiently.
