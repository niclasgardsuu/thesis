
%%% Local Variables:
%%% mode: latex
%%% TeX-master: "main"
%%% End:

In this section, we review research that has already been done on the subject of ZGC, as well as garbage collection in general. By looking at what has been done previously, the goal is to build up knowledge about what more there is to explore, and how this thesis project will build upon what they have discovered. 

\subsection{ZGC}
As ZGC is the main topic of this thesis, and is the reason this thesis exists, it is only fitting to cover some of the previous research that has been done to improve ZGC. The first paper is one from 2019 about \textit{Improving relocation performance in ZGC by identifying the size of small objects}, written by Jinyu Yu~\cite{zgc:yu}. This paper researched the option of reducing the number of relocations being done for compaction by introducing a new classification for object sizes in ZGC. The results from this paper show that it is a valuable effort to try to reduce the number of relocations being done. With a decrease of about 40\% of all relocations, the throughput stayed unchanged during most benchmarks used for evaluation. This thesis will also explore a new way of relocating objects, but define a new way of representing the relocation destination.

Another paper was written by L. Shoravi, where he explored the option of compressing pointers in ZGC~\cite{zgc:shoravi}. The mission of compressing pointers is to reduce the amount of memory used by the garbage collector. While this thesis does not use compression to reduce the amount of memory used by the garbage collector, the goal is to use memory more efficiently and reduce fragmentation, which is also a way of reducing the total memory used by the garbage collector.

\subsubsection{Other Garbage Collectors}
C. Tauro et. al. wrote a paper on the comparison of two different garbage collectors in Java: CMS and G1~\cite{gc:tauro}. CMS, in contrast to the garbage collector studied in this thesis, ZGC, uses free-lists to represent available memory in the heap. In the paper, they find that CMS performs significantly better in memory usage than what is observed by looking at G1, another regional garbage collector. However, according to their evaluations, G1 performs marginally better regarding throughput. This suggests that they both have benefits that can be desirable when running certain programs. This further motivates the work of this thesis, which is to combine the regional characteristics of ZGC and G1, and the usage of a free-list based allocator like in CMS, and look at what can be achieved when they are used together.