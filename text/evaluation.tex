In this section I will explain the different methods used for evaluating the implementation, and how conclusions will be drawn from certain evaluations.

\subsection{Choosing an allocator}
In order to evaluate the performance of the free list based allocator inside of ZGC, an interface for the available allocators must be constructed. By switching between different allocators, it allows for testing the performance when using allocators with different properties and strengths.

During this thesis, two different allocators are used. The first one is an adapted TLSF Allocator~\footnote{TLSF} designed and implemented by Joel Sikström at Oracle. The second allocator is a Buddy Allocator~\footnote{Buddy} designed and implemented by Casper Norrbin at Oracle. The construction of both allocators were done during the time of this thesis project, which allowed us to collaborate and agree upon a common interface for the allocators.

\subsection{Benchmarking}
In order to evalaute the performance of Java running with my new implementation of ZGC, I used a benchmarking framework called Dacapo. Dacapo is a benchmarking tool that allows for the evaluation of Java programs, and is used by many other projects. The framework is designed to be easy to use, and is able to report on the statistics of a program being run, which will be used to evaluate the performance of the new allocator.

%%% Local Variables:
%%% mode: latex
%%% TeX-master: "main"
%%% End: