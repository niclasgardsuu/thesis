\begin{figure}
\begin{center}
  \includegraphics[height=5\baselineskip]{logo_name.pdf}
\hspace{2cm}
  \includegraphics[height=5\baselineskip]{logo_name_sv.pdf}
\end{center}
\caption{Till vänster syns universitetets logotyp i färg, till höger i svartvitt.}
  \label{fig:exempel}
\end{figure}

Beskriv strukturen både internt (hur ert eget system är uppbyggt) och externt (vilka andra system ert system kommunicerar med). \textbf{Använd figurer} (och text)!
\begin{itemize}
\item Vilka delar består systemet av? (T.ex. databas, webbinterface, AI-modul, grafik...) Vilka kommunicerar med vilka, beror av vilka, innehåller vilka andra?
\item Vilka delar fanns färdiga att använda/anpassa, vilka utvecklade ni själva? Visa tydligt, gärna grafiskt.
\item Finns olika alternativa byggblock eller designval? Vilka är argumenten för/emot valen?
\item Hur kommunicerar delarna, vilka protokoll och/eller dataformat används? (Beskriv mer detaljerat i senare, i Huvuddelen.)
\item Finns det olika typer av användare/motsv? (T.ex. administratörer resp slut\-an\-vän\-dare?)
\end{itemize}

\subsection{Tänk på följande}

Var inte för tekniskt detaljerade här.  Tanken är att ge en översikt över systemet.  Ni behöver inte beskriva objektmetoder etc. i detalj (om de inte är nya och avgörande för resultatet). Tekniska detaljer och implementation beskriver ni snarare i Huvudddelen.

Se till att ni använder \emph{samma terminologi} i figurer som visar systemet som i texten. 

\emph{Anknyt figurerna till texten} på ett tydligt sätt. Om ni t.ex. har separata underrubriker som beskriver olika delar/aspekter av systemstrukturen med tillhörande figur, välj antingen en underrubrik per del i figuren eller använd helt andra underrubriker.  Annars kommer läsaren att undra var underrubriken som beskriver del X är, när det finns underrubriker för alla andra delar.


%%% Local Variables:
%%% mode: latex
%%% TeX-master: "rapport-mall"
%%% End:
