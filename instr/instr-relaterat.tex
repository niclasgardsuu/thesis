Här beskriver ni liknande system eller projekt, och förklarar hur de relaterar till ert.  Alltså: vad vet ni om läget när det gäller ``det större problemet'' som projektet ska lösa?  Vilka andra har försökt lösa liknande/närliggande problem, eller gjort relaterade/liknande saker/system? Referera! (Se Appendix~\ref{sec:referenser} för mer om hur.)

Ha alltid ett första stycke som beskriver vilka typer av relaterade arbeten ni tar upp.  Förklara/motivera varför \emph{just dessa} beskrivs och hur ni valt ut dem, så att läsaren övertygas om att ni gjort ett vettigt urval av relaterat arbete (och inte bara beskriver de första google-träffarna). %

Liksom för bakgrunden kan relaterat arbete också gå längre tillbaka. Det är inte nöd\-vän\-digt\-vis bara datorbaserade/appbaserade/etc lösningar som är relaterade.

Några olika typer av relaterat arbete:
\begin{itemize}
\item Direkt relaterade till ert system (t.ex. alternativ till ert system) – och hur är de relaterade?
\item Relaterade till era metoder (t.ex. olika sätt att göra positionering)
\item Relaterade till syftet/grundproblemet (vad ska man använda t.ex. positioneringen till?)
\end{itemize}

Tänk också på:

\begin{itemize}
\item 
  Relaterat arbete bör vara på en generell (gärna akademisk) nivå och inte bara relaterat till en uppdragsgivare, en programmeringsplattform, eller ett särskilt sätt att angripa problemet.
\item 
  När ni jämför ert system med andra, se till att läsaren fått en översikt över vad ert system är först (t.ex. i inledningen) så att vederbörande kan göra en kvalificerad bedömning.
\item
 Beskriv vad varje relaterat arbete är (t.ex. en app, en undersökning\ldots), vad deras resultat var, \textbf{och hur det relaterar till ert arbete}.
\end{itemize}

(Ovanstående är ungefär max-storlek på saker i en punktlista -- är det mer text är det oftast bättre med riktiga paragrafer.)

Ofta är det bra att gruppera relaterade arbeten (t.ex. appar som löser liknande problem, eller andra angreppsätt än tekniska).
Ofta är det effektivt att efter en grupp relaterade arbeten summera hur de relaterar till ert (t.ex. ``dessa appar har dessa liknande finesser, men ingen av dem hanterar X som är en av våra huvudpoänger'').


Efter att ha läst det här avsnittet ska läsaren förstå att ni satt er in i tidigare arbete, och t.ex. inte uppfunnit hjulet på nytt.

%%% Local Variables:
%%% mode: latex
%%% TeX-master: "rapport-mall"
%%% End:
